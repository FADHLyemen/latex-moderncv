%% start of file `template.tex'.
%% Copyright 2006-2010 Xavier Danaux (xdanaux@gmail.com).
%% Copyright 2010-2011 Mark Liu (markwayneliu@gmail.com).
%
% This work may be distributed and/or modified under the
% conditions of the LaTeX Project Public License version 1.3c,
% available at http://www.latex-project.org/lppl/.

\documentclass[11pt,letterpaper,serif]{moderncv}

\usepackage{verbatim}

% moderncv themes
\moderncvstyle{classic}
\moderncvcolor{blue}

\usepackage{lastpage}

% character encoding
\usepackage[utf8]{inputenc}                   % replace by the encoding you are using

% \usepackage[style=numeric-comp,
% sorting=none,defernumbers=true]{biblatex}%mod.
% \DeclareBibliographyCategory{cited}
% \AtEveryCitekey{\addtocategory{cited}{\thefield{entrykey}}}
% \addbibresource{publications.bib}

% adjust the page margins
\usepackage[scale=0.8]{geometry}
%\setlength{\hintscolumnwidth}{3cm}						% if you want to change the width of the column with the dates
%\AtBeginDocument{\setlength{\maketitlenamewidth}{6cm}}  % only for the classic theme, if you want to change the width of your name placeholder (to leave more space for your address details
%\AtBeginDocument{\recomputelengths}                     % required when changes are made to page layout lengths

% --- Begin use with Multibib ----
\usepackage{multibib}
\newcites{book,article,poster}{{Books},{Journal Articles},{Conference Posters}}
% --- END use with Multibib ---

% BibTeX - don't print the year for publications that are "in press"
\newcommand{\dontprintyear}[1]{}


% personal data
\firstname{Olga}
\familyname{Botvinnik}
\title{Curriculum Vitae}                               % optional, remove / comment the line if not wanted

\address{University of California, San Diego}{2880 Torrey Pines Scenic Dr}{La Jolla, CA 92122}    % optional, remove the line if not wanted
% \mobile{(omitted for web)}                    % optional, remove the line if not wanted
\homepage{http://olgabotvinnik.com}                % optional, remove the line if not wanted
\email{olgabot@ucsd.edu}                      % optional, remove the line if not wanted
\social[twitter]{olgabot}                             % optional, remove / comment the line if not wanted
\social[github]{olgabot}                              % optional, remove / comment the line if not wanted
% \date{\today}
\quote{Last Updated: \today}                                 % optional, remove / comment the line if not wanted

%\extrainfo{\url{http://markliu.me}} % optional, remove the line if not wanted

% to show numerical labels in the bibliography; only useful if you make citations in your resume
%\makeatletter
%\renewcommand*{\bibliographyitemlabel}{\@biblabel{\arabic{enumiv}}}
%\makeatother

% Could never get the page numbers to work so added them by hand with \rfoot
% \nopagenumbers{}                             % uncomment to suppress automatic page numbering for CVs longer than one page
%----------------------------------------------------------------------------------
%            content
%----------------------------------------------------------------------------------
\begin{document}

% \input{filename}

\maketitle
\rfoot{\addressfont\itshape\textcolor{gray}{Page \thepage\ of \pageref{LastPage}}}


\cvline{}{Research Interests: Molecular and cellular heterogeneity of biological systems}


\section{Education}
\cventry{2012--Present}{Ph.D. Candidate, Bioinformatics and Systems Biology}{University of California, San Diego}{La Jolla, CA}{}{Thesis: Computational analysis of single-cell alternative splicing. Advisor: Gene Yeo} % arguments 3 to 6 can be left empty
\cventry{2012}{M.S., Bioinformatics and Biomolecular Engineering}{University of California, Santa Cruz}{Santa Cruz, CA}{}{Advisor: Nader Pourmand}
% \cvline{advisor:}{\small Professor Nader Pourmand}
\cventry{2010}{S.B., Biological Engineering}{Massachusetts Institute of Technology}{Cambridge, MA}{}{}
\cventry{2010}{S.B., Mathematics}{Massachusetts Institute of Technology}{Cambridge, MA}{}{}
% \cvline{gpa:}{\small 3.76/4.0}
% \cvline{honors:}{\small Cum Laude}


\section{Research Positions}

\cventry{2013--Present}{Gene Yeo Laboratory}{University of California, San Diego}{La Jolla, CA}{}{Collaborated with wet-lab researchers to analyze single-cell motor neuron differentiation mRNA-seq data. Independently developed several software packages written in Python for alternative splicing analyses}

\cventry{2012--2013}{Research Rotations}{University of California, San Diego}{La Jolla, CA}{}{Worked in Profs. Trey Ideker, Gene Yeo, and Pavel Pevzner’s laboratories}

\cventry{2012}{Nader Pourmand Laboratory}{University of California, Santa Cruz}{Santa Cruz, CA}{}{Developed pipeline to analyze RNA-Seq data, applied to single-cell analysis of breast cancer drug resistance to paclitaxel}

\cventry{2010--2011}{Jill Mesirov Laboratory}{Broad Institute of Harvard and MIT}{Cambridge, MA}{}{Created REVEALER algorithm to unveil candidate oncogenic activators}

\cventry{2010}{Sebastian Seung Laboratory}{MIT Department of Brain and Cognitive Sciences}{Cambridge, MA}{}{Computed directionality of neurons in electron microscopy of rabbit retina inner plexiform layer slices}

\cventry{2009}{David Gifford Laboratory}{MIT Computer Science and Artificial Intelligence Laboratory}{Cambridge, MA}{}{Tested whether measures of information flow can predict gene lethality in different genomic networks}

\cventry{2008}{Sean Eddy Laboratory}{Howard Hughes Medical Institute Janelia Farm Research Campus}{Ashburn, VA}{}{Improved protein homology search by creating a better null homology model with Hidden Markov Models}

\cventry{2007}{Martha Bulyk Laboratory}{Brigham and Women's Hospital, Division of Genetics}{Boston, MA}{}{Analyzed DNA binding specificities of mouse homeodomain transcription factors}


\section{Honors and Awards}

\cvline{2016}{100 Awesome Women In The Open-Source Community You Should Know, sourced.com}
\cvline{2014}{NumFocus John Hunter Technical Fellowship for Open Source Science}
\cvline{2013--2016}{National Defense Science and Engineering Graduate Fellowship}
\cvline{2013}{Fannie and John Hertz Foundation Fellowship Finalist}
\cvline{2012}{National Science Foundation Graduate Research Fellowship: Honorable Mention}
\cvline{2012}{University of California Regents Scholarship}
\cvline{2009}{Bernard M. Gordon-MIT Engineering Leadership Program}
\cvline{2008}{Howard Hughes Medical Institute Janelia Farm Research Summer Scholar}


% --- START For use with BibLatex --- 
% \begin{refsection}[publications]
% \nocite{*} 
% \printbibliography[resetnumbers=true,sorting=ynt,%mod
% title={Entire publication list sorted by year}]   
% \end{refsection}
% ---- END for use with biblatex ----

% --- START For use with BibTeX, not BibLaTeX .... ---
% \nocite{*}
% \bibliographystyle{habbrvyr}
% % \biblographystyle{plain}
% \bibliography{publications}                    % 'publications' is the name of a BibTeX file

% % \nocite{*}  
% % \bibliographystyle{habbrvyr}
% % \bibliography{books}
% ---- END for use with bibtex ----


% --- START for use with multibib package ---
% Publications from a BibTeX file using the multibib package
\section{Publications}

% \nocitebook{Compeau:woKZa7_-}
\nocitearticle{*}
\bibliographystylearticle{habbrvyrolgabold}
\bibliographyarticle{articles}                   % 'articles' is the name of a BibTeX file

\nocitebook{*}
\bibliographystylebook{habbrvyrolgabold}
\bibliographybook{books}                   % 'books' is the name of a BibTeX file

\nociteposter{*}
\bibliographystyleposter{habbrvyrolgabold}
\bibliographyposter{posters}                   % 'posters' is the name of a BibTeX file
% --- END for use with multibib package ----

\section{Talks}

\cventry{2016}{Festival of Genomics California}{San Diego Convention Center}{San Diego, CA}{}{}
\cventry{2016}{Fluidigm User Group Meeting}{City of Hope Hospital}{Los Angeles, CA}{}{}
\cventry{2016}{Bioinformatics and Systems Biology Ph.D. Program Recruitment}{University of California, San Diego}{La Jolla, CA}{}{}
\cventry{2015}{CodeNeuro}{New Museum}{New York, NY}{}{Slides: \url{http://nbviewer.jupyter.org/format/slides/gist/olgabot/2ee1087d74df46c842df#/}}
\cventry{2015}{AmpNeuro}{Amplifying Neuroscience Symposium}{La Jolla, CA}{}{Slides: \url{http://nbviewer.jupyter.org/format/slides/gist/olgabot/ba6970fbfa2babd79f55#/}}
\cventry{2015}{Bioinformatics Exchange}{University of California, San Diego}{La Jolla, CA}{}{}
\cventry{2015}{Bioinformatics and Systems Biology Ph.D. Program Recruitment}{University of California, San Diego}{La Jolla, CA}{}{}
\cventry{2014}{RNA Club}{University of California, San Diego}{La Jolla, CA}{}{}
\cventry{2014}{Bioinformatics EXPO}{University of California, San Diego}{La Jolla, CA}{}{Best Talk, 2nd place}
\cventry{2014}{PyData}{401 Park Ave. South}{New York, NY}{}{}

\section{Teaching, Outreach, and Leadership}

% \cventry{year--year}{Job title}{Employer}{City}{}{Description}

\cventry{2016}{Teaching Assistant}{Cold Spring Harbor Laboratories}{Cold Spring Harbor, NY}{}{
Developed and led bioinformatics coursework of \emph{Single Cell Analysis Course} including alignment, machine learning, Python, and basic command line tools to an audience largely with little to no programming experience. Course materials available at \url{http://github.com/YeoLab/single-cell-bioinformatics}
}

\cventry{2016}{Guest Instructor}{Quantitative Methods in Genetics and Genomics}{La Jolla, CA}{
}{Taught three weeks of \texttt{git}, RNA-seq and analysis methods to graduate-level UCSD course of 30 students, mostly with limited programming experience. Course materials available at \url{http://github.com/biom262/biom262-2016}}

\cventry{2015--2016}{Speaker and Co-Organizer}{CodeNeuro}{New York, NY and San Francisco, CA}{}{
Presented \texttt{flotilla} software, taught ``coding for neuroscientists'' tutorial (\url{http://github.com/codeneuro/gitgoing}), and advanced data analysis tutorial}

\cventry{2015--2016}{President and Co-Founder}{Graduate Bioinformatics Council}{La Jolla, CA}{}{
Founded graduate student council organization for UCSD Bioinformatics and Systems Biology Program. Advocated for student voices, organized ``town hall'' meetings, social hours, fellowhsip peer review, and led a team of eight vice presidents and representatives.
}

\cventry{2013--2016}{Volunteer}{San Diego Science and Engineering Festival}{San Diego, CA}{}{Developed and demonstrated bioinformatics modules to all ages at UCSD Bioinformatics booth.}

\cventry{2013--2014}{Instructor}{Bioinformatics Algorithms}{Coursera.org}{}{Developed interactive curriculum for online Bioinformatics Algorithms Coursera class and textbook. Advisors: Pavel Pevzner and Phillip Compeau}

\cventry{2011--2012}{Mentor}{We Teach Science}{San Jose, CA}{}{Weekly algebra tutoring to an 8th grader}

\cventry{2011--2012}{Guest Instructor}{Pacific Collegiate School}{Santa Cruz, CA}{}{Created bioinformatics modules to engage students in tying genotype to phenotype for high school AP Biology}

\cventry{2012}{Co-Chair}{Intelligent Systems for Molecular Biology Student Council Symposium}{}{Long Beach, CA}{}

\cventry{2012}{Instructor}{Minority Access to Research Careers}{Santa Cruz, CA}{}{Taught inquiry-based stem cell bioinformatics curriculum to undergraduate researchers}
\cventry{2011}{Volunteer}{Science Club for Girls}{Cambridge, MA}{}{Co-led after-school biology science club for a class of 16 2nd graders}

\cventry{2009--2011}{Choreographer}{MIT DanceTroupe}{Cambridge, MA}{}{Taught beginner to intermediate hip-hop choreography to fellow students}
\cventry{2008--2010}{Publicity Chair}{MIT DanceTroupe}{Cambridge, MA}{}{Designed posters and T-shirts to publicize and promote DanceTroupe concert attendance}
\cventry{2008}{Social Chair}{Baker House}{Cambridge, MA}{}{Organized social events for students, including a popular ``Dormal'' event with catered dinner and jazz performances}


\section{Software}
\cvline{}{All software is written in Python and open source, licensed under the 3-clause BSD license.}
\cvline{\texttt{anchor}}{Categorizes alternative splicing data into ``modes''---bimodal, unimodal, or uniform. \url{http://github.com/YeoLab/anchor}}
\cvline{\texttt{bonvoyage}}{Transforms 1d splicing profiles into 2d space to maximize interpretability of change in signal. \url{http://github.com/YeoLab/bonvoyage}}

\cvline{\texttt{flotilla}}{All-in-one package to perform machine learning analyses on large-scale molecular profiling datasets such as gene expression and alternative splicing. \url{http://github.com/YeoLab/flotilla}}

\cvline{\texttt{kvector}}{Counts $k$-mers in DNA or RNA as $k$-mer vectors, transforms position weight matrices (PWMs) to $k$-mer vectors. \url{http://github.com/olgabot/kvector}}

\cvline{\texttt{outrigger}}{Fast \emph{De novo} alternative exon detection and quantification. \url{http://github.com/YeoLab/outrigger}}

\cvline{\texttt{poshsplice}}{Annotates alternative splicing events with biological features such as translated protein product. \url{http://github.com/olgabot/poshsplice}}

\cvline{\small\texttt{prettyplotlib}}{Painlessly create beautiful \texttt{matplotlib} plots. \url{http://github.com/olgabot/prettyplotlib}}

\cvline{\texttt{seaborn}}{Contributor, wrote clustered heatmap classes and function. \url{http://github.com/mwaskom/seaborn}}

\end{document}
